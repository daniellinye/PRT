\documentclass{article}
\usepackage[margin=1.5cm]{geometry} %Feel free to adjust this margin to your liking
\usepackage[utf8]{inputenc} %So we can use special characters
\usepackage{graphicx} %You might need this to include an image
\usepackage{amsmath} %For various math functions..
\usepackage{hyperref} %For creating hyperlinks

%opening
\title{Database assignment 2}
\author{Daniel Lin and Robert Arntzenius}

\begin{document}

\maketitle

\section{Relational Algebra}

\subparagraph*{Exercize 1}
-

\chapter{1.}


\pi_{\text{mid}}
\sigma_{\text{"title" = "Not clickbait"} \wedge \text{"likecount"} > 9000 }messages


\chapter{2.}


\rho(invited, \pi_{\text{id}}((\pi_{\text{eid}} \sigma_{\text{name="Cheap sunglasses check description"}} event)\bowtie_{{eid}} 
invitedToEvent))\bowtie_{\text{id}} user)

\rho(users, \pi_{\text{id}}user)

invited \bowtie_{\text{invited.id $ \neq $ user.id}} users

\chapter{3.}

\rho(mes1, message)

\rho(mes2, message)

\rho(res, \pi_{\text{mes1.mid}} ( (message \times message) - \sigma_{\text{mes1.likecount} < \text{mes2.likecount}}(mes1 \times mes2) ) )

\pi_{\text{id}}(\pi_{\text{mid}}(res)\bowtie_{\text{mid}}messageLikes)\bowtie_{\text{id}}user

\chapter{3.}




\sigma_{\text{"title" = "Not clickbait"} \wedge \text{"likecount"} > 9000 }users



\end{document}
