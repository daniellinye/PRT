\documentclass{article}
\usepackage[margin=1.5cm]{geometry} %Feel free to adjust this margin to your liking
\usepackage[utf8]{inputenc} %So we can use special characters
\usepackage{graphicx} %You might need this to include an image
\usepackage{amsmath} %For various math functions..
\usepackage{hyperref} %For creating hyperlinks

\title{Databases 2018 - \LaTeX\ Example}
\author{Leiden Institute of Advanced Computer Science}
%\date{01-01-1971} %Automatically uses date of compiling unless specified.

%The actual document starts here..
\begin{document}
\maketitle %This command uses the data from above to create a title heading
This is a short example document to help you create a submission written using \LaTeX\ for your second assignment. It is essentially an abbreviated source for the assignment document itself. \\

Students can use the online \LaTeX\ compiler Overleaf: \url{https://www.overleaf.com/}, or install a LaTeX compiler locally, for example Windows users could use MiKTeX for a basic installation: \url{https://miktex.org/download}. 

\section{Relational Algebra}
You are part of a legitimate data analysis institution that has acquired legitimate data from a certain social media platform. You have received the following schemas:
\begin{itemize}
	\item user(\underline{id}, name, gender, age):
		\begin{itemize}
		\item id: unique integer identifying the user
		\item name: string with the user's full name
		\item gender: string that contains the user's gender
		\item age: the age of the user
		\end{itemize}
	\item invitedToEvent(\underline{id}, \underline{eid}, date, accepted):
		\begin{itemize}
		\item id: foreign key referencing user.id
		\item eid: foreign key referencing event.eid
		\item date: integer with same formatting as friends.date. Shows when user was invited to the event
		\item accepted: boolean that shows whether user has accepted invitation or not
		\end{itemize}
\end{itemize}
%Above itemization is indented for readability only, it is not required

\begin{enumerate}
	\item Find the mids of messages that have been liked 9000 times or more, and have the title "Not clickbait".
	\item ...
	\item Find the id of the creator(s) of the most liked message(s).
\end{enumerate}

\section{Schema Normalization}
You have been approached to help with the design of a database for a video sharing platform. They currently store their data about videos and comments in a single large table. They want to store the following data in their database:
\begin{itemize}
	\item \underline{V}ideo ID
	\item ...
	\item Comment \underline{A}uthor
\end{itemize}
The above underlinings show suggested abbreviations to be used later in this exercise.\\

They have identified that there is a lot of redundant data in their database right now. Because of this their website is running very slowly and they are running out of server space.\\

They have given you the following statements, some of which might indicate one or more functional dependencies.
\begin{itemize}
	\item A video consists of an ID, Title, Uploader, Category, Subcategory and a Description.
	\item ...
	\item The Body of a comment may not contain any emoji.
\end{itemize}
Currently, the table \texttt{R: VTUCSDNBA} stores the information in the database. For each of the following questions, \emph{show your work} and \emph{explain your answers}. Showing your work is more important than having a correct answer.
\begin{enumerate}
	\item Determine all the functional dependencies (F) that are derivable from the points mentioned above.
	\item ...
	\item Prove, if $XY\rightarrow B$ and $X \rightarrow Y$ then $X \rightarrow B$.
\end{enumerate}

\section{Transaction Management}
{\small %you can try \tiny font size here for a smaller table
\begin{tabular}{|l||lllllllllllllll|}
\hline
T1:&R(C)& & & & &R(B)& &W(C)& & & & &W(A)&&Cmt.\\ \hline
T2:&& &R(B)& &W(B)& & & & &W(B)& &R(A)& &W(C)&Cmt.\\ \hline
T3:&&R(A)& &W(A) & & & & &R(A)& & & & &&Cmt.\\ \hline
T4:&& & & & & &R(C) & & & &W(A)& & &&Cmt.\\ \hline
\end{tabular}
}

\begin{enumerate}
	\item Draw the precedence graph of the above schedule.
	\item Is the above conflict serializable? Explain your answer.
	\item Apply Strict 2PL to the above schedule and determine the waits-for graph for the schedule right after the first deadlock occurs.
\end{enumerate}

\appendix %indicates that we start the appendix section of the document
\section{\LaTeX\ Reference}
It is recommended to briefly browse the simple introduction to \LaTeX\ in the wikibook, specifically the Mathematics section thereof: \url{https://en.wikibooks.org/wiki/LaTeX/Mathematics} \\

We put an inline math formula between \$. For example, \verb-$x^2+2x=3$- will become $x^2+2x=3$. If we want to give the math formula more space we put it between double \$. For example \verb+$$f(x)=\delta_{0}-x^i$$+ Will become $$f(x)=\delta_{0}-x^i$$

For Schema Normalization it will probably be enough to know the \verb+$\rightarrow$+: $\rightarrow$. In this way we can make simple derivations of functional dependencies, like this:
\begin{verbatim}
\begin{align*} 
A &\rightarrow B \\
B &\rightarrow C \\
A &\rightarrow C \ \text{(Transitivity)}
\end{align*}
\end{verbatim}
\begin{align*} %the * here indicates unnumbered equations
A &\rightarrow B \\
B &\rightarrow C \\
A &\rightarrow C \ \text{(Transitivity)}
\end{align*}
The align environment is very useful for multi-lined mathematics, such as what you might be doing with schema normalization.

For Relational Algebra a simple answer to a query might look like this:
\begin{verbatim}
\pi_{\text{name}}\sigma_{\text{age} > 18}\text{Students}
\end{verbatim}
$$
\pi_{\text{name}}\sigma_{\text{age} > 18}\text{Students}
$$
Here the underscore (\_) denotes the use of sub-script.
Also, note the use of \verb+\text{}+ for words. This command is from the \emph{amsmath} package. If we didn't use it, \LaTeX\ would interpret all written characters as mathematical variables and not typeset them correctly.\\

Other commands that might be of use for you while writing your Relational Algebra queries:\\

\begin{tabular}{|p{.55\textwidth} | p{.4\textwidth}|}
\hline
\begin{verbatim}
A \bowtie B
\end{verbatim} & $$A \bowtie B$$\\
\hline
\begin{verbatim}
A \times B
\end{verbatim}
&
$$
A \times B
$$
\\
\hline
\begin{verbatim}
(\text{age}>18 \vee \text{age} < 65) 
\wedge \text{name}=\text{"Hans"}
\end{verbatim}
&
$$
(\text{age}>18 \vee \text{age} < 65) \wedge \text{name}=\text{"Hans"}
$$
\\
\hline
\begin{verbatim}
\pi\ \sigma\ \rho\
\end{verbatim} &
$$\pi\ \sigma\ \rho\ $$ \\
\hline
\begin{verbatim}
S\bowtie_{\text{sid}}R
\end{verbatim} &
$$S\bowtie_{\text{sid}}R$$ \\
\hline
\begin{verbatim}
\begin{align*}
\rho(X_1, \pi_{\text{name}}S) \\
X_1 \bowtie R
\end{align*}
\end{verbatim} &
\begin{align*}
\rho(X_1, \pi_{\text{name}}S) \\
X_1 \bowtie R
\end{align*} \\
\hline
\begin{verbatim}
A \setminus B, A / B
\end{verbatim} & $$A \setminus B, A / B$$\\
\hline
\end{tabular} \\

For Transaction Management we recommend using the tabular environment as described here: \url{https://en.wikibooks.org/wiki/LaTeX/Tables}


\end{document}
