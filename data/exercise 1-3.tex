\documentclass{article}
\usepackage[margin=1.5cm]{geometry} %Feel free to adjust this margin to your liking
\usepackage[utf8]{inputenc} %So we can use special characters
\usepackage{graphicx} %You might need this to include an image
\usepackage{amsmath} %For various math functions..
\usepackage{hyperref} %For creating hyperlinks

%opening
\title{Database assignment 2}
\author{Daniel Lin and Robert Arntzenius}

\begin{document}

\maketitle

\section{Relational Algebra}

\subparagraph*{Exercize 1}
-

\chapter{1.}


\pi_{\text{mid}}
\sigma_{\text{"title" = "Not clickbait"} \wedge \text{"likecount"} > 9000 }messages


\chapter{2.}


\rho(invited, \pi_{\text{id}}((\pi_{\text{eid}} \sigma_{\text{name="Cheap sunglasses check description"}} event)\bowtie_{{eid}} 
invitedToEvent))\bowtie_{\text{id}} user)

\rho(users, \pi_{\text{id}}user)

\pi_{\text{id}}(invited \bowtie_{\text{invited.id $ \neq $ user.id}} users)

\chapter{3.}

\rho(mes1, message)

\rho(mes2, message)

\rho(res, \pi_{\text{mes1.mid}} ( (message \times message) - \sigma_{\text{mes1.likecount} < \text{mes2.likecount}}(mes1 \times mes2) ) )

\pi_{\text{id}}(\pi_{\text{mid}}(res)\bowtie_{\text{mid}}messageLikes)\bowtie_{\text{id}}user

\chapter{4.}

\rho(invited, \pi_{\text{id}}(\sigma_{\text{accepted = "true"}}(\pi_{\text{eid}} \sigma_{\text{name="Nude painting"}} event)\bowtie_{{eid}} 
invitedToEvent))\bowtie_{\text{id}} user)

\rho(users, \pi_{\text{id}}user)

%grab friends
\rho(friends, \pi_{\text{id}}(\pi_{fid}friend)\bowtie_{id}user)


\pi_{\text{name}}(invited \bowtie_{\text{invited.id = user.id}} users \cup friends)



\chapter{5.}

\rho(friends, \pi_{\text{id}}(\pi_{fid}friend)\bowtie_{id}user)

\rho(othergender, \pi_{\text{id}}friends - \pi_{\text{id}}(\pi_{\text{id}}friends)\bowtie_{\text{gender}} user)

%this might not work yet
\rho(otherage, \pi_{\text{id}}(\pi_{\text{id}}friends)\bowtie_{\text{friends.age} < \text{user.age}}user)

\pi_{\text{name}}(friends/(othergender \cap otherage))user


\chapter{6.}

\rho(one, \pi_{\text{id}}\sigma_{\text{name = "Crazy Cosplay Caroline"}}user)

\rho(onemessages, \pi_{\text{mid}}\rho_{\text{creator=one}}message)

%grab people that have liked the message
\rho(messageliked, \pi_{\text{id}}\sigma_{\text{mid=onemessages}}messageLikes)

%give names
\pi_{\text{name}}((messageliked)/(\pi_{\text{id, name}}user))


\chapter{7.}


%tip was to use cartesian product
%or self-joins
\rho(mes1, \sigma_{likecount > 999}message)

\rho(mes2, \sigma_{likecount > 999}message)

\rho(mes3, \sigma_{likecount > 999}message)

\rho(res1, \pi_{\text{mes1.creator}} (\sigma_{\text{mes1.mid} \neq \text{mes2.mid} \wedge \text{mes2.mid} \neq \text{mes3.mid} \wedge \text{mes1.mid}  \neq \text{mes3.mid}}(mes1 \times mes2 \times mes3) ) )

\rho(res2, \pi_{\text{mes2.creator}} (\sigma_{\text{mes1.mid} \neq \text{mes2.mid} \wedge \text{mes2.mid} \neq \text{mes3.mid} \wedge \text{mes1.mid}  \neq \text{mes3.mid}}(mes1 \times mes2 \times mes3) ) )

\rho(res3, \pi_{\text{mes3.creator}} (\sigma_{\text{mes1.mid} \neq \text{mes2.mid} \wedge \text{mes2.mid} \neq \text{mes3.mid} \wedge \text{mes1.mid}  \neq \text{mes3.mid}}(mes1 \times mes2 \times mes3) ) )


\pi_{\text{id}}(\sigma_{\text{res1.creator} = \text{res2.creator} \wedge \text{res2.creator} = \text{res3.creator} \wedge \text{res1.creator}  = \text{res3.creator}}(res1 \times res2 \times res3)})\bowtie_{res.creator=user.id}user




\end{document}
